% Personalisation des chapitres
\titleformat{\chapter}
[hang] % Type du chapitre (valeur par défaut)
{\Huge\bfseries} % Taille et style du numéro du chapitre
{Annexe \thechapter % Afficher le numéro du chapitre
\hspace{0.3cm} % Espace horizontal entre numéro et séparateur
\textcolor{gray75}{\textbar} % Séparateur entre numéro et titre : pipe
\hspace{0.3cm}}{0cm} % Espace horizontal entre séparateur et chapitre
{\Huge\bfseries} % Type et style du titre

% Réduction de l'espace au dessus et en dessous des chapitres
\titlespacing{\chapter}{0pt}{-10pt}{20pt}

% Permet d'écrire 'Annexe A' dans la table des matières
\begin{appendices}

\begin{landscape}

\chapter{Organismes de cybersécurité}
\label{annexeA} % Permet de faire un lien avec \ref dans le texte

Les organismes de cybersécurité en France et en Europe.

\begin{figure}[H] % H pour here
    \centering
    \includegraphics[width=0.80\paperwidth]{imgo.jpeg}
    \caption{Possibilité de mettre une légende}
\end{figure}

\end{landscape}

\end{appendices}
