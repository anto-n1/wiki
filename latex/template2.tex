% --------------------------------------------------------------------
% RAPPORT
%
% Début d'écriture du document : décembre 2020
% --------------------------------------------------------------------

% --------------------------------------------------------------------
% Nomenclature et règles d'écriture du document
%
% - Longueur max d'une ligne : 80 caractères
% - Mettre des commentaires
% - Include des différentes parties pour éviter taille immense de fichier
% --------------------------------------------------------------------

% --------------------------------------------------------------------
% Compilation du document
%
% xelatex document
% xelatex document
% bibtex document
% pdflatex document
% pdflatex document
%
% Avex TeXworks sous Windows :
% Utiliser la compilation avec Xelatex+MakeIndex+BibTeX
% --------------------------------------------------------------------

% --------------------------------------------------------------------
% Préambule
% --------------------------------------------------------------------

% Type du document
\documentclass[a4paper,french,11pt]{report}

% Encodage (permet de taper les accents sans balises dans le texte notamment)
\usepackage[utf8]{inputenc}

% Manières de la langue française pour LaTeX
\usepackage[french]{babel}

% Justification du texte
\usepackage{microtype}

% Insertion d'images et répertoire où les trouver
\usepackage{graphicx}
\graphicspath{{images/}}

% Faire des listes à puce
\usepackage{enumitem}

% Faire des listes sans tirets/bulles alignées (page acteurs de ma formation)
\usepackage{scrextend}

% Modification des tailles de la marge
\usepackage[left=2cm,right=2cm,top=2.7cm,bottom=2.7cm]{geometry}

% Utiliser des légende pour les images et les tableaux
\usepackage{caption}

% Ajouter des pages et/ou fichiers PDF externes
\usepackage{pdfpages}

% Permet d'utiliser \lastpage pour afficher le numéro de la dernière page
\usepackage{lastpage}

% Ajouter un espace blanc (ligne blanche) entre les paragraphes
% Pas besoin d'utiliser '\\' pour ajouter une ligne blanche
\setlength{\parskip}{1em}

% Taille de l'alinéa début de paragraphe
% Même espacement que les sous-sous-titres
\setlength{\parindent}{2.5em}

% Gestion des espaces entre les lignes (interligne)
%\renewcommand{\baselinestretch}{2}\normalsize

% Pour les double quotes \enquote{mon texte}
\usepackage{csquotes}

% Gestion des liens dans le document
\usepackage{hyperref}
\hypersetup{
    colorlinks=true, % Colorer les liens
    citecolor=black,
    filecolor=black,
    linkcolor=black, % Couleur des liens
    urlcolor=black,
    hyperfigures=true % Ajouter liens du sommaire vers les parties lors du clic
}

% Bibliographie
\usepackage[
    %backend=biber, % Choix de l'outil de traitement
    backend=bibtex,
    babel=hyphen, % Env linguistique dédié aux entrées en langues alternatives
    backref=true, % Liens dans la bibliographie pour remonter dans le texte
    backrefstyle=none, % Afficher toutes les utilisations de la référence
    bibstyle=numeric, % Style pour les clés des références dans la bibliographie
    citestyle=numeric, % Style pour les clés des références dans le texte
    sorting=none % Bibliographie triée par ordre d'utilisation des références
]{biblatex} % Support des bibliographies

% Nom du fichier contenant les sources
%\addbibresource{sources.bib}
\bibliography{sources.bib}

% Réaliser des annexes
\usepackage[titletoc]{appendix}

% --------------------------------------------------------------------
% Police d'écriture
% --------------------------------------------------------------------

% Dans le cas de l'utilisation de pdflatex
% Dans ce cas, compiler avec pdflatex et non xelatex
%\usepackage{lmodern}

% Utilisation de xelatex pour utiliser calibri
\usepackage{fontspec}
\setmainfont{Carlito} % Linux
%\setmainfont{Helvetica} % macOS
%\setmainfont{Calibri} % Windows

% --------------------------------------------------------------------
% Définition des couleurs
% --------------------------------------------------------------------

% Ajouter la possibilité d'utiliser de la couleur pour le texte
\usepackage{xcolor}

\definecolor{gray75}{gray}{0.75}

% --------------------------------------------------------------------
% Personalisation des titres
% --------------------------------------------------------------------

\usepackage{titlesec}

 % Personalisation des chapitres
\titleformat{\chapter}
[hang] % Type du chapitre (valeur par défaut)
{\Huge\bfseries} % Taille et style du numéro du chapitre
{\thechapter % Afficher le numéro du chapitre
\hspace{0.3cm} % Espace horizontal entre numéro et séparateur
\textcolor{gray75}{\textbar} % Séparateur entre numéro et titre : pipe
\hspace{0.3cm}}{0cm} % Espace horizontal entre séparateur et chapitre
{\Huge\bfseries} % Type et style du titre

% Réduction de l'espace au dessus et en dessous des chapitres
\titlespacing{\chapter}{0pt}{-30pt}{20pt}

% --------------------------------------------------------------------
% Personalisation des sous-titres
% --------------------------------------------------------------------

\titleformat{\section}
[hang] % Type du chapitre (valeur par défaut)
{\Large\bfseries} % Taille et style du numéro du chapitre
{\thesection % Afficher le numéro du sous-titre
\hspace{0.15cm} % Espace horizontal entre numéro et séparateur
\textcolor{gray75}{-} % Séparateur entre numéro et titre : pipe
\hspace{0.15cm}}{0cm} % Espace horizontal entre séparateur et chapitre
{\Large\bfseries} % Type et style du titre

\titlespacing{\section}{0em}{15pt}{15pt}

% Utilisation des lettres majuscules pour classer les sous-titres
\renewcommand*{\thesection}{\Alph{section}}

% --------------------------------------------------------------------
% Personalisation des sous-sous-titres
% --------------------------------------------------------------------

\titleformat{\subsection}
[hang] % Type du chapitre (valeur par défaut)
{\large\bfseries} % Taille et style du numéro du chapitre
{\thesubsection % Afficher le numéro du sous-titre
\hspace{0.20cm} % Espace horizontal entre numéro et séparateur
\textcolor{gray75}{-} % Séparateur entre numéro et titre : pipe
\hspace{0.15cm}}{0cm} % Espace horizontal entre séparateur et chapitre
{\large\bfseries} % Type et style du titre

\titlespacing{\subsection}{2.5em}{10pt}{10pt}

% --------------------------------------------------------------------
% Header et footer première page
% --------------------------------------------------------------------

\usepackage{fancyhdr}
\fancypagestyle{first}{

% Effacer les headers et footer par défaut
\fancyhf{}

% Suppression des lignes horizontales
\renewcommand{\headrulewidth}{0pt}
\renewcommand{\footrulewidth}{0pt}
}

% --------------------------------------------------------------------
% Header et footer pages de titre et de chapitres
% --------------------------------------------------------------------

\fancypagestyle{plain}{

% Effacer les headers et footer par défaut
\fancyhf{}

% Suppression des lignes horizontales
\renewcommand{\headrulewidth}{0pt}
\renewcommand{\footrulewidth}{0pt}
}

% --------------------------------------------------------------------
% Header et footer pages de texte
% --------------------------------------------------------------------

\pagestyle{fancy}

% Effacer les headers et footer par défaut
\fancyhf{}

\fancyhead[L]{Projet de fin d'études}
% Ecrire uniquement les noms des chapitres sans le mot "chapter" avec \leftmark
\renewcommand{\chaptermark}[1]{\markboth{#1}{#1}}
% \thecapter avec des accolades pour respecter l'espace avant le tiret
\fancyhead[R]{\thechapter{} - \leftmark}

\fancyfoot[L]{DOUILLARD Antonin}
\fancyfoot[R]{\thepage/\pageref{LastPage}} % Affichage du type 1/5

% Lignes horizontales
\renewcommand{\headrulewidth}{1pt}
\renewcommand{\footrulewidth}{1pt}

% --------------------------------------------------------------------
% Header et footer supprimés pour les pages non souhaitées
% --------------------------------------------------------------------

\fancypagestyle{none}{

% Effacer les headers et footer par défaut
\fancyhf{}

% Suppression des lignes horizontales
\renewcommand{\headrulewidth}{0pt}
\renewcommand{\footrulewidth}{0pt}
}

% --------------------------------------------------------------------
% Création d'environnements
% --------------------------------------------------------------------

% Désactiver l'interligne de paragraphe défini avec \setlength{\parindent}
\newenvironment{nospaceline}{
	\setlength{\parskip}{1pt}
}

% --------------------------------------------------------------------
% Début du rapport
% --------------------------------------------------------------------

\begin{document}

% --------------------------------------------------------------------
% Page de titre
% --------------------------------------------------------------------

% Lignes horizontales utilisées pour le titre
\newcommand{\HRule}[1]{\rule{\linewidth}{#1}}

\begin{titlepage}

% Footer du type chapitre et titres sur la première page
\thispagestyle{first}

\begin{center}

\vspace{0.4cm}

\LARGE{\uppercase{Un titre}}

\vspace{0.5cm}

\Large{Un premier sous titre}

\vspace{1cm}

\HRule{0.75pt} \\
\vspace{0.7cm}

\huge{\textbf{\uppercase{titre principal}}}

\HRule{0.75pt} \\

\vspace{1.5cm}

\LARGE{Apprenti : \textbf{Auteur}}

\vspace{0.6cm}

\Large{Quelque chose}

\Large{Février 2020}

\vspace{1.5cm}

\includegraphics[width=5cm]{image.png}

\includegraphics[width=5cm]{image.png}

\end{center}

\end{titlepage}

% --------------------------------------------------------------------
% Table des matières, table des figures
% --------------------------------------------------------------------

% Permet de désactiver temporairement l'interligne
\begin{nospaceline}

% Table des matières
\tableofcontents

% Table des figures
%\listoffigures

\end{nospaceline}

% --------------------------------------------------------------------
% Corps du rapport
% --------------------------------------------------------------------

% Introduction sans numérotation
\chapter*{Introduction} % Ne pas mettre de numéro de chapitre
\addcontentsline{toc}{chapter}{Introduction} % Afficher dans le sommaire
\chaptermark{Introduction}{} % Afficher le nom du chapitre dans le header

% Include des différentes parties
\input{01_introduction.tex}
\input{02_partie_une.tex}

\chapter{Partie}

Corps du texte.

\section{Section}

Corps du texte.

\newpage

Page numéro 2, permettant de voir le footer.

% --------------------------------------------------------------------
% Lexique
% --------------------------------------------------------------------

\chapter*{Lexique}
\addcontentsline{toc}{chapter}{Lexique}
\chaptermark{Lexique}{} % Afficher le nom du chapitre dans le header

% --------------------------------------------------------------------
% Ressources externes - exemples pour les citations
% --------------------------------------------------------------------

\chapter*{Ressources externes}
\addcontentsline{toc}{chapter}{Ressources externes}
\markboth{Ressources externes}{} % Afficher le nom du chapitre dans le header

\cite[cf.][Ce que je veux afficher]{nomDeLaSource}

% --------------------------------------------------------------------
% Sources
% --------------------------------------------------------------------

% Afficher la bibliographie avec comme titre Bibliographie et la mettre dans la
% table des matières
\printbibliography[heading=bibintoc, title={Bibliographie}]

% --------------------------------------------------------------------
% Annexes
% --------------------------------------------------------------------

% Personalisation des chapitres
\titleformat{\chapter}
[hang] % Type du chapitre (valeur par défaut)
{\Huge\bfseries} % Taille et style du numéro du chapitre
{Annexe \thechapter % Afficher le numéro du chapitre
\hspace{0.3cm} % Espace horizontal entre numéro et séparateur
\textcolor{gray75}{\textbar} % Séparateur entre numéro et titre : pipe
\hspace{0.3cm}}{0cm} % Espace horizontal entre séparateur et chapitre
{\Huge\bfseries} % Type et style du titre

% Réduction de l'espace au dessus et en dessous des chapitres
\titlespacing{\chapter}{0pt}{-10pt}{20pt}

% Permet d'écrire 'Annexe A' dans la table des matières
\begin{appendices}

\begin{landscape}

\chapter{Organismes de cybersécurité}

Les organismes de cybersécurité en France et en Europe.

\begin{figure}[h]
    \centering
    \includegraphics[width=0.80\paperwidth]{imgo.jpeg}
\end{figure}

\end{landscape}

\end{appendices}

% --------------------------------------------------------------------
% Fin du rapport
% --------------------------------------------------------------------

\end{document}
