% --------------------------------------------------------------------
% RAPPORT XXX
%
% Ce document est...
%
% DOUILLARD Antonin
% ESAIP IRA 4
%
% Début d'écriture du document : décembre 2020
%
% - include des différentes parties pour éviter taille immense de fichier
% --------------------------------------------------------------------

% --------------------------------------------------------------------
% Nomenclature et règles d'écriture du document
%
% - Longueur max d'une ligne : 90 caractères
% - Mettre des commentaires
% --------------------------------------------------------------------

% --------------------------------------------------------------------
% Compilation du document
%
% pdflatex document.tex
% pdflatex document.tex
% biber document
% pdflatex document.tex
% pdflatex document.tex
% --------------------------------------------------------------------

% --------------------------------------------------------------------
% Préambule
% --------------------------------------------------------------------

% Type du document
\documentclass[a4paper,french,11pt]{report}

% Encodage (permet de taper les accents sans balises directement dans le texte notamment)
\usepackage[utf8]{inputenc}

% Manières de la langue française pour LaTeX
\usepackage[french]{babel}

% Choix de la police d'écriture
\usepackage{lmodern}

% Justification du texte
\usepackage{microtype}

% Insertion d'images et répertoire où les trouver
\usepackage{graphicx}
\graphicspath{{images/}}

% Faire des listes à puce
\usepackage{enumitem}

% Faire des listes sans tirets/bulles alignées (page acteurs de ma formation)
\usepackage{scrextend}

% Modification des tailles de la marge
\usepackage[left=2cm,right=2cm,top=2.7cm,bottom=2.7cm]{geometry}

% Ajouter la possibilité d'utiliser la couleur pour le texte
\usepackage{xcolor}

% Utiliser des légende pour les images et les tableaux
\usepackage{caption}

% Utiliser biblatex (biber) pour la bibliographie
\usepackage[backend=biber]{biblatex}
% Nom du fichier contenant les sources
\addbibresource{sources.bib}

% --------------------------------------------------------------------
% Personalisation des titres
% --------------------------------------------------------------------

\usepackage{titlesec}

 % Personalisation des chapitres
\titleformat{\chapter}
[hang] % Type du chapitre (valeur par défaut)
{\Huge\bfseries} % Taille et style du numéro du chapitre
{\thechapter % Afficher le numéro du chapitre
\hspace{0.3cm} % Espace horizontal entre numéro et séparateur
\textcolor{gray75}{\textbar} % Séparateur entre numéro et titre : pipe
\hspace{0.3cm}}{0cm} % Espace horizontal entre séparateur et chapitre
{\Huge\bfseries} % Type et style du titre

% Réduction de l'espace au dessus et en dessous des chapitres
%\titlespacing{command}{left spacing}{before spacing}{after spacing}[right]
\titlespacing{\chapter}{0pt}{-30pt}{40pt}

% --------------------------------------------------------------------
% Header et footer pages de texte
% --------------------------------------------------------------------

\usepackage{fancyhdr}
\pagestyle{fancy}

% Effacer les headers et footer par défaut
\fancyhf{}

\fancyhead[L]{Projet de fin d'études}
\fancyhead[R]{\leftmark}

\fancyfoot[R]{\thepage}
\fancyfoot[L]{DOUILLARD Antonin}
\fancyfoot[C]{\fbox{TEXTE ENCADRE}}

% Lignes horizontales
\renewcommand{\headrulewidth}{1pt}
\renewcommand{\footrulewidth}{1pt}

% --------------------------------------------------------------------
% Header et footer pages de titre et de chapitres
% --------------------------------------------------------------------

\fancypagestyle{plain}{

% Effacer les headers et footer par défaut
\fancyhf{}
\fancyfoot[C]{\fbox{TEXTE ENCADRE}}

% Suppression des lignes horizontales
\renewcommand{\headrulewidth}{0pt}
\renewcommand{\footrulewidth}{0pt}
}

% --------------------------------------------------------------------
% Début du rapport
% --------------------------------------------------------------------

\begin{document}

% --------------------------------------------------------------------
% Page de titre
% --------------------------------------------------------------------

\begin{titlepage}

% Footer du type chapitre et titres sur la première page
\thispagestyle{plain}

\begin{center}
\Huge{\textbf{Projet de fin d'études}}
\vspace{0.5cm}
\LARGE{SUJET}
           
           
\includegraphics[width=0.4\textwidth]{logo_1.png}
\includegraphics[width=0.4\textwidth]{logo_2.png}
           
\end{center}

\end{titlepage}

% --------------------------------------------------------------------
% Table des matières, table des figures
% --------------------------------------------------------------------

% Table des matières
\tableofcontents

% Table des figures
\listoffigures

% --------------------------------------------------------------------
% Page présentant les acteurs de ma formation
% --------------------------------------------------------------------

\chapter{Les acteurs de ma formation}

% Liste alignée
\begin{labeling}{alligator}

\item [\textbf{Tuteur industriel}] likes bananas
\item [\textbf{Tuteur industriel}] very dangerous animal, sharp teeth, long

\end{labeling}

% --------------------------------------------------------------------
% Corps du rapport
% --------------------------------------------------------------------

\chapter{Partie}

Corps du texte.

\newpage

Page numéro 2 ,permettant de voir le footer.

% --------------------------------------------------------------------
% Sources
% --------------------------------------------------------------------

% Afficher la bibliographie avec comme titre Bibliographie et la mettre dans la table des
% matières
\printbibliography[heading=bibintoc,title={Bibliographie}]

% --------------------------------------------------------------------
% Fin du rapport
% --------------------------------------------------------------------

\end{document}
