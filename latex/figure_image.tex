\usepackage{float} % Permet d'utiliser le [H]

% Insertion d'images et répertoire où les trouver
\usepackage{graphicx}
\graphicspath{{images/}}

% Utiliser des légende pour les images et les tableaux
\usepackage{caption}


% Deux images côte à côte centrées
\begin{figure}[H]
    \centering
    \begin{tabular}{cc}
    \includegraphics[height=1.7cm]{image1.png} \hspace{0.3cm} & \hspace{0.3cm} \includegraphics[height=1.7cm]{image2.png}
    \end{tabular}
\end{figure}

% Image centrée
\begin{figure}[H]
    \centering
    \includegraphics[width=5cm]{image.png}
\end{figure}

% Image centrée avec légende
% H pour "here", permet de placer l'image à l'endroit où elle est déclarée,
% sinon LaTeX la place là où il trouve de la place
\begin{figure}[H]
    \centering
    \includegraphics[width=5cm]{image.png}
    \caption{Légende de l'image}
\end{figure}
